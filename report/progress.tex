\documentclass[a4paper]{article}

\usepackage[norsk]{babel}
\usepackage{chicago}
\usepackage[T1]{fontenc}
\usepackage{footnote}
\usepackage{multirow}

\title{Latisk morfologi f\"ur idioten\footnote{\emph{ie}.~datamaskiner}}
\author{Arne Skj\ae{}rholt}

\begin{document}
\maketitle

\section{Verbet --- Del 1}

\begin{table}[hp]
\begin{center}
\begin{tabular}{|c|c|c|c|}
    \hline
       & \multicolumn{2}{c|}{Presens} & \multirow{2}{*}{Perfektum} \\ \cline{2-3}
       & Aktiv & Passiv & \\
    \hline
1.~sg. & \emph{--m}   & \emph{--r}    & \emph{--i}       \\
2.~sg. & \emph{--s}   & \emph{--ris}  & \emph{--isti}    \\
3.~sg. & \emph{--t}   & \emph{--tur}  & \emph{--it}      \\
1.~pl. & \emph{--mus} & \emph{--mur}  & \emph{--imus}    \\
1.~pl. & \emph{--tis} & \emph{--mini} & \emph{--istis}   \\
1.~pl. & \emph{--nt}  & \emph{--ntur} & \emph{--erunt}   \\
    \hline
\end{tabular}
\caption{De latinske verbendelsene}
\label{endelser}
\end{center}
\end{table}

Tabell \ref{endelser} oppsummerer de eneste, nesten, verbalendelsene man
trenger \aa{} kunne for \aa{} b\o{}ye latinske verb. I tillegg til tabellen
kommer f\o{}rste person entall av presens indikativ som har et eget sett med
endelser, \emph{--o} og \emph{--or} i henholdsvis aktiv og passiv, og en liten
h\aa{}ndfull med idiosynkratiske endelser.

En fullverdig latinsk verbform dannes ved at man tar korrekt form av verbet
som b\o{}yes, legger til et suffiks som angir hvilken tid og modus vi befinner
oss i, og riktig endelse fra tabellen. Tabell \ref{tempus} gir
tempussuffiksene; tallene i parenteser angir hvilke konjugasjoner som bruker
suffikset.

\begin{savenotes}
\begin{table}[hp]
\begin{center}
\begin{tabular}{|c|c|c|}
    \hline
                  & Indikativ   & Konjunktiv \\
    \hline
Presens           & $\emptyset$ & \emph{--e} (1), \emph{--a} (2, 3, 4) \\
Imperfektum       & \emph{--ba} & \emph{--re} \\
Futurum           & \emph{--b}\footnote{Med dette suffikset brukes presens
indikativ-endelsene} (1, 2), \emph{--a} (3, 4) & --- \\
    \hline
Perfektum         & $\emptyset$\footnote{Med perfektumsendelsene} & \emph{--eri} \\
Pluskvamperfektum & \emph{--era} & \emph{--isse} \\
Futurum exactum   & \emph{--eri}\footnote{I 1.~sg.~er hele endelsen
\emph{--ero}} & ---        \\
    \hline
\end{tabular}
\caption{Tempussuffiksene}
\label{tempus}
\end{center}
\end{table}
\end{savenotes}

N\aa{}r vi s\aa{} har tatt verbalstammen og lagt til tempussuffiks og
personsuffiks til er vi halvveis til den endelige formen vi er p\aa{} jakt
etter. Hvis vi for eksempel er p\aa{} jakt etter 2.~pl.~pluskvamperfektum
konjunktiv av verbet \emph{amo} (representert som
\texttt{amo+Verb+Pqp+Subj+2P+Act} i implementasjonen). Perfektumstammen av
\emph{amo} er \emph{amav--}, og kombinert med tempus-- og personsuffiksene
f\aa{}r vi formen \emph{amav--isse--tis}.

Det siste vi m\aa{} gj\o{}re for \aa{} f\aa{} det \o{}nskede resultatet er
\aa{} utf\o{}re de n\o{}dvendige fonetiske endringene. Det er seks
p\aa{}krevde transformasjoner og fire valgfrie. De p\aa{}krevde reglene er:
\begin{enumerate}
\item \emph{a} foran morfemgrense etterfulgt av \emph{o} eller \emph{e} faller
bort.

\item Morfemgrense med konsonant foran og etterf\o{}lgende \emph{r} realiseres
som \emph{e}.

\item Kort \emph{i} foran morfemgrense etterfulgt av \emph{r} realiseres som
\emph{e} (denne regelen tilsvarer regelen over, men for 3B--konjugasjonen).

\item Kort \emph{i} foran endelsen \emph{--i} faller bort (slik at presens
passiv infinitiv av \emph{capio} er \emph{capi}, ikke \emph{capii}).

\item Morfemgrense mellom konsonant og \emph{s}, \emph{t} eller \emph{m}
realiseres som \emph{i}.

\item Morfemgrense mellom konsonant eller \emph{i} og \emph{n} realiseres som
\emph{u}.
\end{enumerate}

% XXX: Flytte dette til Del 2?
De valgfrie reglene\footnote{Strengt tatt \'en valgfri regel og noen som er
p\aa{}krevd i tilfellet hvor den valgfrie transformasjonen er utf\o{}rt}
h\aa{}ndterer former i perfektumsystemet med bortfall av \emph{--v}.

\begin{itemize}
\item \emph{v} foran morfemgrense, med \emph{a}, \emph{i} eller \emph{e} foran
og etterf\o{}lgende vokal og \emph{s} eller \emph{r} hvis vokalen foran er
\emph{a}, kan sl\o{}yfes.

\item \emph{i} etter morfemgrense med sl\o{}yfet \emph{v} foran skal ogs\aa{}
sl\o{}yfes hvis vokalen foran \emph{v} var \emph{a} eller hvis den var
\emph{i} og \emph{i}en som sl\o{}yfes er etterfulgt av \emph{s}.

\item \emph{e} etter morfemgrense med sl\o{}yfet \emph{v} skal ogs\aa{}
sl\o{}yfes hvis vokalen foran \emph{v} er \emph{a}.

\item \emph{v} kan ikke sl\o{}yfes hvis vokalen foran er \emph{a} og endelsen
slutter p\aa{} \emph{re} (forhindrer \emph{amav--ere} $\to$ \emph{amare}).
\end{itemize}

Disse reglene i form av toniv\aa{}regler er listet i tabell \ref{twolc}.

\begin{savenotes}
\begin{table}[hp]
\begin{center}
\begin{tabular}{rcl}
\texttt{a:0  } & \texttt{<=>} & \texttt{\_ \%-: [ o | e ]} \\
\texttt{\%-:e} & \texttt{<=>} & \texttt{Cons\footnote{\texttt{Cons} er en konsonant eller \emph{qu} } \_ r} \\
\texttt{I:e  } & \texttt{<=>} & \texttt{\_ \%-: r} \\
\texttt{I:0  } & \texttt{<=>} & \texttt{\_ \%-: i .\#.} \\
\texttt{\%-:i} & \texttt{<=>} & \texttt{Cons \_ [ s | t | m ]} \\
\texttt{\%-:u} & \texttt{<=>} & \texttt{[ Cons | :i ] \_ n} \\
\texttt{v:0  } & \texttt{=> } & \texttt{a \_ \%-: [ i: | e: ] [ s | r ]} \\
\texttt{     } & \texttt{   } & \texttt{i \_ \%-: [ i: | e: ]} \\
\texttt{     } & \texttt{   } & \texttt{e \_ \%-: [ i: | e: ]} \\
\texttt{v:0  } & \texttt{/<=} & \texttt{a \_ \%-: ? r e .\#.} \\
\texttt{i:0  } & \texttt{<=>} & \texttt{a v:0 \%-: \_} \\
\texttt{     } & \texttt{   } & \texttt{i v:0 \%-: \_ s} \\
\texttt{e:0  } & \texttt{<=>} & \texttt{a v:0 \%-: \_} \\
\end{tabular}
\caption{De fonetiske toniv\aa{}reglene}
\label{twolc}
\end{center}
\end{table}
\end{savenotes}

\section{Verbet --- Del 2}
Den forrige seksjonen tar seg av de fleste formene i verbalparadigmet, men det
er et par ting som gjenst\aa{}r. For det f\o{}rste er det et lite knippe
endelser som forekommer som klammeformer: i stedet for \emph{--ris} i
2.~person entall passiv kan formen \emph{--re} forekomme, og i 3.~person
flertall perfektum indikativ kan vi ha \emph{--ere} i stedet for
\emph{--erunt}.

Men mer presserende er det kanskje at det er enkelte former som mangler for at
verbalparadigmet skal v\ae{}re komplett: inifinitiver og imperativer (og
supinum og patisipper, men de er ikke implementert enn\aa{}). Tabellene
\ref{infinitiv} og \ref{imperativ} gir endelsene for henholdsvis infinitivene
og imperativene.

\begin{table}[hp]
\begin{center}
\begin{tabular}{|c|c|c|}
    \hline
          & Aktiv        & Passiv \\
    \hline
Presens   & \emph{--re}  & \emph{--i} \\
Perfektum & \emph{--isse} & --- \\
    \hline
\end{tabular}
\caption{Infinitivsendelser}
\label{infinitiv}
\end{center}
\end{table}

\begin{table}[hp]
\begin{center}
\begin{tabular}{|c|c|c|}
    \hline
       & Presens & Futurum \\
    \hline
2.~sg. & $\emptyset$ & \emph{--to} \\
3.~sg. & ---        & \emph{--to} \\
2.~pl. & \emph{--te} & \emph{--tote} \\
3.~pl. & ---         & \emph{--nto} \\
    \hline
\end{tabular}
\caption{Imperativendelser}
\label{imperativ}
\end{center}
\end{table}

Disse endelsene, kombinert med de fonetiske reglene fra del 1 gir et komplett
bilde av det latinske verbet.

\bibliographystyle{chicago}
\bibliography{report}{}

\end{document}
