\documentclass{article}

%\usepackage[norsk]{babel}
\usepackage{cite}

\title{Two-level morphology for Latin}
\author{Arne Skj\ae{}rholt}

\begin{document}
\maketitle
% XXX: Move TOC (and stuff like list of figures/tables if they crop up as
% well) to the end of the document instead of the top?
\tableofcontents

\section{Introduction}
\subsection{Purpose}
Implementing two-level morphology is no simple task (see \cite{bondihoved}), so the purpose of this
project is to implmement a subset of Latin morphology that is as interesting
as possible. Hopefully this will include at least nominal, verbal and
pronominal morphology.

\subsection{The plan}
The plan of attack is as follows: first read up on the tools to be used and
the theory involved. Then, start with implementing a subset of verbal
morphology; this will probably be the active present system of regular verbs
of the first class, then the perfect system. Experiences from this should give
some ideas on how to structure different inflectional stems for a single
lexical entry, and from this it will hopefully be relatively easy to implement
regular nominal morphology. After this we may either proceed to pronominal or
adjectival morphology, or return to the verbal morpholgy and add more verb
classes, passives and non-finite forms of the verb.

Certain factors may alter this plan, however. The current corpus should be
consulted to generate statistics of which parts of the morphology are more
frequent and thus more likely to be useful to implement first. Also, the users
of the system should be asked about which parts they would prefer to be
implemented first.
% TODO: Actually do this!

\section{The Implementation}
\subsection{Tools used}

\begin{itemize}
    \item XFST \cite{xfst}
    \item Lewis and Short from Perseus for lexicon data? % TODO: Find reference
\end{itemize}

\section{Conclusion} % TODO: Better title!

% TODO: Find references for grammars. Eitrem, Ernout & Thomas, the other one. The Swedish one.
\bibliographystyle{plain} % TODO: Better citation style?
\bibliography{report}{}

\end{document}
