\documentclass{article}

\usepackage[norsk]{babel}
\usepackage{chicago}
\usepackage[T1]{fontenc}
\usepackage{footnote}
\usepackage{multirow}
\usepackage{verbatim}

\newcommand\code[1]{\texttt{#1}}
\newcommand\form[1]{\texttt{#1}} % XXX: Endre til \texttt?
\newcommand\note[1]{\marginpar{\raggedright\tiny#1}}
\newcommand\prog[1]{\texttt{#1}}
\newcommand\w[1]{\emph{#1}}

\title{Toniv\aa{}morfologi for latin}
\author{Arne Skj\ae{}rholt}

\begin{document}
\maketitle

\section{Introduksjon}
\subsection{Form\aa{}l}
Det \aa{} implementere en komplett toniv\aa{}modell er ikke en triviell
oppgave, det er blant annet skrevet hovedfagsoppgaver der m\aa{}let er nettopp
det, se \citeN{bondihoved} og \citeN{french}, s\aa{} m\aa{}let for dette
prosjektet har v\ae{}rt \aa{} implementere en delmengde av det latinske
formverket som er s\aa{} interessant som mulig; der interessant er definert
slik at s\aa{} mange implementasjonstekniske problemer som mulig blir
eksponert. Uregelmessige former, selv om de kanskje rent praktisk sett er
veldig interessante, er for eksempel ikke s\ae{}rlig inspirerende i denne
sammenhengen ettersom implementasjonen stort sett kan oppsummeres som ``for
denne roten ser denne formen slik ut''.

\subsection{Planen}
Den planlagte fremgangsm\aa{}ten, som stort sett korresponderer med den
faktiske gangen i arbeidet med implementasjonen, var \aa{} begynne i det
sm\aa{} med regelrette substantiver for \aa{} bli kjent med verkt\o{}yene og
gi grunnleggende f\o{}ringer p\aa{} hvordan de kan integreres med standard
UNIX-verkt\o{}y som \texttt{make} for \aa{} automatisere
kompilasjonsprosessen. Etter dette sto verbalsystemet for tur. Siden
verbalparadigmet er den mest kompliserte delen av den latinske morfologien
ans\aa{} jeg det som viktig \aa{} utforske denne delen av systemet for \aa{}
finne ut hvor kompliserte regler som trengtes. Etter dette var planen \aa{}
implementere det som gjensto av de regelrette verbal-- og
substantivparadigmene.

\subsection{Funksjonalitet}\note{Dette m\aa{} skrives.}
Systemet analyserer og genererer verb-- og substantivformer, for leksikonet
gitt i tabell \ref{leksikon}. For substantiver skal systemet ha full dekning
for regelrette ord i alle fem deklinasjoner \note{Trippelsjekke at dette er
s\aa{} riktig som mulig!}; likeledes skal systemet klare alle regelrette
adjektiver i alle kj\o{}nn, kasus og grader, samt adverbene avledet av dem i
alle tre grader. Systemet skal ogs\aa{} ha full dekning p\aa{} regelrette
verb, b\aa{}de finitte og infinitte former samt verbaladjektiver. Til grunn
for forml\ae{}ren ligger stort sett paradigmene og reglene fra \citeN{eitrem},
supplert med oppklarende detaljer fra \citeN{ernout}.

\begin{table}[hp]
\begin{center}
\begin{tabular}{|ccc|}
    \hline
Substantiv  & Adjektiv   & Verb \\
    \hline
\w{rosa}    & \w{longus} & \w{amo}     \\
\w{dominus} & \w{acer}   & \w{hortor}  \\
\w{numerus} & \w{brevis} & \w{moneo}   \\
\w{filius}  & \w{felix}  & \w{vereor}  \\
\w{puer}    &            & \w{rego}    \\
\w{ager}    &            & \w{sequor}  \\
\w{bellum}  &            & \w{capio}   \\
\w{rex}     &            & \w{patior}  \\
\w{corpus}  &            & \w{audio}   \\
\w{fructus} &            & \w{partior} \\
\w{cornus}  &            &             \\
\w{res}     &            &             \\
    \hline
\end{tabular}
\end{center}
\caption{Leksikon}
\label{leksikon}
\end{table}

\section{Implementasjonen}
% TODO: Flette dette inn andre steder i teksten!
%\subsection{Kilder}\note{Bedre tittel! Eller bare sl\o{}yfe hele avsnittet og
%flette dette inn andre steder i teksten?}
%Grunnlaget for forml\ae{}ren er i all hovedsak tatt fra de relevante
%paragrafene i \citeN{eitrem}, men med en del detaljer fylt ut av
%\citeN{ernout}; sistnevnte var s\ae{}rlig opplysende i enkelte detaljer
%omkring perfektumsformer med synkope av \w{--v}.

\subsection{Verkt\o{}yene}\note{Ikke forn\o{}yd med dette avsnittet. Det er
ting som burde sies, men s\aa{}nn teksten flyter rundt dette n\aa{} er det
ikke bra nok\ldots}
Selve analysatoren er skrevet med Xerox-verkt\o{}ykassen for endelig
tilstandsmorfologi, dokumentert i \citeN{xfst}; prosjektet benytter seg av
\texttt{lexc} til leksikon og \texttt{twolc} for fonetiske regler.
\texttt{lexc} spesifiserer regul\ae{}re relasjoner som h\o{}yrerekursive
grammatikker og \texttt{twolc} spesifiserer relasjoner i form av tegn som er
relatert, og i hvilke kontekster de kan, m\aa{} eller ikke kan opptre.

Slik Xerox-verkt\o{}yene fungerer er prosjektet b\aa{}de en morfologisk
analysator og generator, ettersom alle nettverkene som produseres kan
kj\o{}res i begge retninger: b\aa{}de fra en leksikonform som
\form{amo+Verb+Pres+Subj+2S+Pass} og fra en overflateform som \w{ameris}.

\subsection{Bruk av systemet}
Beskrive kort hvordan systemet brukes. Ta utgangspunkt i web-grensesnittet.

Et webgrensesnitt mot applikasjonen er tilgjengelig p\aa{}
\texttt{http://heim.ifi.uio.no/arnskj/cgi-bin/latin/analyse.cgi}.

\subsection{Nominalsystemet}
\label{deklinasjon}
Den latinske nominalmorfologien er forholdsvis enkel. I motsetning til for
eksempel nominalsystemet i sanskrit er det ingen lydendringer innad i
b\o{}yningsr\o{}ttene, og de f\aa{} tilfellene av lydendringer i m\o{}tet
mellom rot og endelse er enkle \aa{} hanskes med. Dette gir en oppbygning der
hver rot er bundet mot en b\o{}yningsklasse med sine karakteristiske endelser
for de enkelte formene, samt \'en fonetisk regel for \aa{} h\aa{}ndtere
r\o{}tter p\aa{} \w{-er} som mister \w{e} n\aa{}r det legges til en endelse.

For \aa{} lette implementasjonen har de fleste vekslinger av overflateformen
til roten blitt ansett for \aa{} v\ae{}re leksikalske (unntaket er bortfall av
\w{e} i \w{-er}). Rent lingvistisk hadde det kanskje v\ae{}rt riktigere \aa{}
modellere flere av disse fenomenene som regelrette med fonetiske regler, for
eksempel \w{reg-s} > \w{rex} og \w{flos-is} > \w{floris}. P\aa{} samme
m\aa{}te ender stammefinal vokal opp som endelsesinitial i f\o{}rste og fjerde
deklinasjon, slik at bortfallet av vokalen i dativ/ablativ flertall kan
h\aa{}ndteres i leksikon i stedet for med fonetiske regler. Som nevnt er den
ene fonetiske regelen i nominalsystemet for \aa{} h\aa{}ndtere
synkope\note{Synkope er riktig, no?} i \w{-er}-endelsen. Den fungerer ganske
enkelt slik at de r\o{}ttene som mister \w{e} med endelse har denne markert
som s\aa{}dan i leksikon og den fonetiske regelen fjerner den hvis det kommer
en endelse etter; opprinnelig var denne regelen mer generell, og synkoperte
alle forekomster av \w{-er} foran morfemgrense (unntatt i r\o{}ttene \w{puer},
\w{gener}, \w{socer}, \w{liber}, \w{vesper} og \w{signifer}), men det viser
seg at det finnes r\o{}tter p\aa{} \w{-er} som \emph{ikke} har synkope, for
eksempel \w{numerus}, og at synkopen er leksikalisert (se for\o{}vrig
\ref{diakroni} for mer diskusjon om dette).

Adjektivb\o{}yningen h\aa{}ndteres p\aa{} akkurat samme m\aa{}te som
substantivene, inkludert \w{e}-synkope. Derimot er avledningen av adverb noe
utradisjonelt implentert.\note{Finne ut riktig m\aa{}te \aa{} formulere dette
p\aa{}\ldots}


%Det viser seg at nominalmorfologien er enkel nok til at det holder \aa{} ha et
%leksikon med r\o{}tter koblet til b\o{}yningsklasse der de forskjellige
%b\o{}yningsklassene deler ut egnede endelser til de forskjellige kasus.
%\note{Dette er ikke sant lenger. F\aa{} med reglene for \aa{} h\aa{}ndtere
%\w{ager} et al.}
%
%Rent lingvistisk hadde kanskje en tiln\ae{}rming lik den som er benyttet for
%konjugasjonen (se seksjon \ref{konjugasjon}) v\ae{}rt riktigere, men for \aa{}
%lette implementasjonen (og for min sjelefred) er en forenklet metode
%foretrukket. Den viktigste konsekvensen av dette er at enhver form for
%veksling mellom b\o{}yningsstammen i de oblike kasus og formen i nominativ
%entall i tredje deklinasjon er ansett som leksikalsk, selv om den kan anses
%som regelmessig, \`a la \w{reg-s} > \w{rex}. P\aa{} samme m\aa{}te ender
%stammefinal vokal opp som endelsesinitiell vokal i stedet i
%vokaldeklinasjonene p\aa{} grunn av stammevokalens bortfall i enkelte former;
%for eksempel er stammen til \w{rosa} ansett for \aa{} v\ae{}re \w{ros--} slik
%at endelsen i dat/ablativ flertall kan v\ae{}re \w{--is} uten flere regler for
%\aa{} h\aa{}ndtere tapet av den stammefinale \w{--a}. \note{Faa med
%adjektiver, adverb (med nota om hacken for aa faa riktige tagger) og
%partisipper.}

\subsection{Verbalsystemet}
\label{konjugasjon}
Verbalmorfologien er den mest interessante og kompliserte delen av systemet;
den er delt inn i to deler: en leksikondel, som inneholder mesteparten av
koden, og et sett fonetiske omskrivningsregler. Reglene i verbleksikonet tar
leksikonformer og skriver dem om til r\o{}tter etterfulgt av en streng
morfemer, for eksempel er resultatet for \w{amo} presens konjunktiv 2.~person
entall passiv \w{ama-e-ris}. Denne mellomformen behandles s\aa{} av de
fonetiske reglene som setter inn korrekte temavokaler og h\aa{}ndterer
bortfall av lyder; for eksempel omskrives \w{ama-o} til \w{amo} og \w{reg-ris}
til \w{regeris}. Tabellene \ref{endelser} og \ref{tempus} gir en oversikt over
de viktigste person-- og tempusendelsene, henholdsvis, som brukes i systemet.

\begin{table}[hp]
\begin{center}
\begin{tabular}{|c|c|c|c|}
    \hline
       & \multicolumn{2}{c|}{Presens} & \multirow{2}{*}{Perfektum} \\ \cline{2-3}
       & Aktiv & Passiv & \\
    \hline
1.~sg. & \w{--m}   & \w{--r}    & \w{--i}     \\
2.~sg. & \w{--s}   & \w{--ris}  & \w{--isti}  \\
3.~sg. & \w{--t}   & \w{--tur}  & \w{--it}    \\
1.~pl. & \w{--mus} & \w{--mur}  & \w{--imus}  \\
1.~pl. & \w{--tis} & \w{--mini} & \w{--istis} \\
1.~pl. & \w{--nt}  & \w{--ntur} & \w{--erunt} \\
    \hline
\end{tabular}
\caption{Personendelsene}
\label{endelser}
\end{center}
\end{table}

\begin{table}[hp]
\begin{center}
\begin{tabular}{|c|c|c|}
    \hline
                  & Indikativ   & Konjunktiv \\
    \hline
Presens           & $\emptyset$ & \w{--e} (1), \w{--a} (2, 3, 4) \\
Imperfektum       & \w{--ba} & \w{--re} \\
Futurum           & \w{--b} (1, 2), \w{--a} (3, 4) & --- \\
    \hline
Perfektum         & $\emptyset$ & \w{--eri} \\
Pluskvamperfektum & \w{--era} & \w{--isse} \\
Futurum exactum   & \w{--eri} & ---        \\
    \hline
\end{tabular}
\caption{Tempussuffiksene}
\label{tempus}
\end{center}
\end{table}

\note{Faa med at 3B verb noteres med -I i leksikon for aa merke dem som
spesielle.}

\note{De fonetiske reglene er oppdatert siden dette ble skrevet. Sjekk mot
verbs-twolc.txt!}
\paragraph{\texttt{a:0 <=> \_ \%-: [ o | e ]}} Denne regelen h\aa{}ndterer to
tilfeller hvor stammefinal \w{--a} faller bort: foran \w{--o} i 1.~person
entall presens indikativ og foran suffikset for presens konjunktiv.

\paragraph{\texttt{\%-:i <=> Cons \_ [ s | t | m ]}} Temavokal i
konsonantstammer er \w{i} foran \w{s}, \w{t} og \w{m}.

\paragraph{\texttt{\%-:e <=> Cons \_ r}} Temavokal i konsonantstammer er
\w{e} foran \w{r}.

\paragraph{\texttt{\%-:u <=> Cons \_ n}} Temavokal i konsonantstammer er
\w{u} foran \w{n}.

\paragraph{\texttt{I:e <=> \_ \%-: r}} Som foreg\aa{}ende regel, men for
stammer p\aa{} kort \w{i}.

\paragraph{\texttt{I:0 <=> \_ \%-: i .\#.}} Stammefinal kort \w{i} faller bort
foran endelsen \w{--i} (presens infinitiv passiv).

Disse reglene beskriver mer eller mindre generelle regler. De resterende
reglene h\aa{}ndterer den valgfrie \w{v}-synkopen i perfektumssystemet.

\paragraph{\texttt{v:0 => \ldots}} \w{v} mellom vokal og morfemgrense kan i
enkelte tilfeller falle bort (se \S 41.3 i \citeN{eitrem} og \citeN{ernout},
s.~209-212).

\paragraph{\texttt{v:0 /<= a \_ \%-: ? r e .\#.}} Bortfall av \w{v} m\aa{}
ikke skje n\aa{}r resultatet vil kunne forveksles med en infinitiv.

\paragraph{\texttt{e:0 <=> a v:0 \%-: \_}} \w{e} etter bortfalt \w{v} foran
\w{a} faller ogs\aa{} bort.

\paragraph{\texttt{i:0 <=> a v:0 \%-: \_ ; i v:0 \%-: \_ s}} \w{i} etter
bortfalt \w{v} foran \w{a}, eller foran \w{i} og etter bortfalt \w{v} foran
\w{i}, faller ogs\aa{} bort.

\subsection{Verifisering}
For \aa{} kontrollere at endringer i en del av koden ikke \o{}delegger noe
annet har jeg skrevet en del tester. Testene er implementert som
Python-programmer som genererer output etter protokollen ``Test Anything
Protocol'' (TAP). Programmet \prog{prove} kj\o{}rer s\aa{} alle
testprogrammene, tolker resultatene og skriver ut statistikker. Testene er
skrevet slik at for hvert par av leksikonform og korrekte former blir de
mulige overflateformene i f\o{}lge reglene generert, og programmet sjekker at:
1) reglene genererer samme antall former som er spesifisert i testprogrammet
og 2) alle formene i testprogrammet blir generert. Totalt blir det utf\o{}rt
5638 tester, for 2892 overflateformer \note{Disse tallene m\aa{} oppdateres!}
(inklusive tilfeller der forskjellige leksikonformer har samme overflateform).
Testene er i stor grad generert ved hjelp av klipp-og-lim og erstatninger ved
hjelp av regul\ae{}re uttrykk.

\section{Synkroni vs.~diakroni}
\label{diakroni}
En morfologisk modell er synkron av natur, men det kan kan ogs\aa{} v\ae{}re
interessant \aa{} se hvordan den synkrone modellen passer med den diakrone
utviklingen av det morfologiske systemet.\note{Ting som m\aa{} med:
\begin{itemize}
\item \w{-er}-synkope. Leksikalisert. Hvorfor?
\end{itemize}
}

\section{Konklusjon}\note{Bedre tittel!}
\note{Ta med ideen om \aa{} simulere historiske regler med
toniv\aa{}formalismen? Det er en riktig festlig ide, og er ikke helt p\aa{}
jordet irrelevant heller, gitt den siste delen med sammenligning av synkroni
og diakroni\ldots}

\clearpage
\bibliographystyle{norchicago}
\bibliography{report}{}

\tableofcontents
\listoftables

\end{document}
