\documentclass{article}

\usepackage[norsk]{babel}
\usepackage{chicago}
\usepackage[T1]{fontenc}
\usepackage{footnote}
\usepackage{multirow}
\usepackage{verbatim}

%\newcommand\code[1]{\texttt{#1}}
%\newcommand\form[1]{\texttt{#1}}
\newcommand\note[1]{\marginpar{\raggedright\tiny#1}}
%\newcommand\prog[1]{\texttt{#1}}
%\newcommand\w[1]{\emph{#1}}

\let\code\texttt
\let\form\texttt
\let\lib\emph
\let\prog\texttt
\let\w\emph

\title{Toniv\aa{}morfologi for latin}
%\author{Kandidat 6472}
%\author{Arne Skj\ae{}rholt}
\date{} % XXX: Ja of nee?

\begin{document}
\maketitle

\section{Introduksjon}
%\subsection{Historisk bakgrunn}
Tradisjonelle generative modeller for morfologi ble skrevet som
kontekst-sensitive omskrivningsregler p\aa{} m\o{}nsteret $\alpha \to
\beta/\gamma \_ \delta$: $\alpha$ omskrives til $\beta$ mellom $\gamma$ og
$\delta$, der $\alpha$, $\beta$, $\gamma$ og $\delta$ er vilk\aa{}rlig
kompliserte strenger. I det generelle tilfellet er slike regler et meget
vanskelig problem \aa{} l\o{}se\footnote{Problemet er det som kalles
PSPACE-komplett. Dette inneb\ae{}rer antageligvis at tiden det tar \aa{}
l\o{}se problemet tar tid som \o{}ker eksponensielt med st\o{}rrelsen p\aa{}
grammatikken. Hvorvidt dette faktisk er tilfellet er et av de store ul\o{}ste
problemene i teoretisk informatikk, men de fleste mener det er tilfelle.}, men
det viser seg at man i generativ morfologi antar at man beveger seg videre i
strengen etter at en omskrivning er utf\o{}rt, slik at n\aa{}r
$\gamma\alpha\delta$ er omskrevet til $\gamma\beta\delta$ forblir
$\beta$-delen uendret, og man fortsetter \aa{} omskrive til h\o{}yre eller
venstre for $\beta$. Dette endrer kompleksiteten p\aa{} problemet radikalt, og
dette problemet viser seg \aa{} kunne l\o{}ses med endelige
tilstandsmaskiner\footnote{Et problem som kan l\o{}ses i polynomisk tid.}.
Dette arbeidet, som begynte p\aa{} 60-tallet, kuliminerte i Kimmo Koskenniemis
toniv\aa{}formalisme i 1983, som fort ble det gjeldende paradigmet i feltet. \cite{twolc}

%\subsection{Form\aa{}l}
Selv om problemet viser seg \aa{} v\ae{}re beregningsmessing forholdsvis lett
\aa{} h\aa{}ndtere, er det \aa{} implementere en komplett toniv\aa{}modell
ikke en triviell oppgave, det er blant annet skrevet hovedfagsoppgaver der
m\aa{}let er nettopp dette, se \citeN{bondihoved} og \citeN{french}. M\aa{}let
for dette prosjektet har derfor v\ae{}rt \aa{} implementere en delmengde av
det latinske formverket som er s\aa{} interessant som mulig; der interessant
er definert slik at s\aa{} mange implementasjonstekniske problemer som mulig
blir eksponert. Uregelmessige former, selv om de kanskje rent praktisk sett er
veldig interessante, er for eksempel ikke s\ae{}rlig inspirerende i denne
sammenhengen ettersom implementasjonen stort sett kan oppsummeres som ``for
denne roten ser denne formen slik ut''.

%\subsection{Planen}
Den planlagte fremgangsm\aa{}ten, som stort sett korresponderer med den
faktiske gangen i arbeidet med implementasjonen, var \aa{} begynne i det
sm\aa{} med regelrette substantiver for \aa{} bli kjent med verkt\o{}yene og
gi grunnleggende f\o{}ringer p\aa{} hvordan de kan integreres med standard
UNIX-verkt\o{}y som \prog{make} for \aa{} automatisere kompilasjonsprosessen.
Etter dette sto verbalsystemet for tur. Siden verbalparadigmet er den mest
kompliserte delen av den latinske morfologien ans\aa{} jeg det som viktig
\aa{} utforske denne delen av systemet for \aa{} finne ut hvor kompliserte
regler som trengtes. Etter dette var planen \aa{} implementere det som gjensto
av de regelrette verbal-- og substantivparadigmene.

%\subsection{Funksjonalitet}\note{Dette m\aa{} skrives.}
I sin n\aa{}v\ae{}rende form analyserer og genererer systemet verb--,
substantiv-- og adjektivformer for leksikonet gitt i tabell \ref{leksikon}.
For substantiv skal applikasjonen ha full dekning for regelrette ord i
f\o{}rste, andre, fjerde og femte deklinasjon. I den tredje deklinasjonen
mangler paradigmene p\aa{} \w{--ium} i genitiv flertall. Systemet skal
ogs\aa{} klare regelrette adjektiver i alle kj\o{}nn og grader, samt deres
avledede adverber. I verbalparadigmet skal det ogs\aa{} v\ae{}re full dekning
for regelrette verb, b\aa{}de vanlige og deponente verb, etter alle de fire
konjugasjonene, inklusive i-stammeverbene i tredje konjugasjon (eksklusive
perifrastiske former).

\begin{table}
\begin{center}
\begin{tabular}{|ccc|}
    \hline
Substantiv  & Adjektiv   & Verb \\
    \hline
\w{rosa}    & \w{longus} & \w{amo}     \\
\w{dominus} & \w{acer}   & \w{hortor}  \\
\w{numerus} & \w{brevis} & \w{moneo}   \\
\w{filius}  & \w{felix}  & \w{vereor}  \\
\w{puer}    &            & \w{rego}    \\
\w{ager}    &            & \w{sequor}  \\
\w{bellum}  &            & \w{capio}   \\
\w{rex}     &            & \w{patior}  \\
\w{corpus}  &            & \w{audio}   \\
\w{fructus} &            & \w{partior} \\
\w{cornus}  &            &             \\
\w{res}     &            &             \\
    \hline
\end{tabular}
\end{center}
\caption{Leksikon}
\label{leksikon}
\end{table}

Vi skal n\aa{} f\o{}rst se p\aa{} hvilken programvare systemet er avhengig av,
hva slags modell som ligger til grunn for implementasjonen, og hvordan selve
implementasjonen er utf\o{}rt. Til slutt skal vi s\aa{} hvordan systemet
passer inn i en mer teoretisk lingvistisk kontekst.

\section{Implementasjonen}
% TODO: Flette dette inn andre steder i teksten!
%\subsection{Kilder}\note{Bedre tittel! Eller bare sl\o{}yfe hele avsnittet og
%flette dette inn andre steder i teksten?}
%Grunnlaget for forml\ae{}ren er i all hovedsak tatt fra de relevante
%paragrafene i \citeN{eitrem}, men med en del detaljer fylt ut av
%\citeN{ernout}; sistnevnte var s\ae{}rlig opplysende i enkelte detaljer
%omkring perfektumsformer med synkope av \w{--v}.
\subsection{Verkt\o{}yene}\note{Ikke forn\o{}yd med dette avsnittet. Det er
ting som burde sies, men s\aa{}nn teksten flyter rundt dette n\aa{} er det
ikke bra nok\ldots}
Selve analysatoren er skrevet med Xerox-verkt\o{}ykassen for endelig
tilstandsmorfologi (XFST), dokumentert i \citeN{xfst}; prosjektet benytter seg av
verkt\o{}yene \prog{lexc} til leksikon og \prog{twolc} for fonetiske regler.
\prog{lexc} spesifiserer regul\ae{}re relasjoner som h\o{}yrerekursive
grammatikker; \prog{twolc} spesifiserer relasjoner i form av kontekstsensitive
grammatikker, men p\aa{} grunn av antagelsen at bare \'en regel kan benyttes
p\aa{} et vilk\aa{}rlig punkt i inputstr\o{}mmen er uttrykkskraften i
grammatikkene begrenset til regul\ae{}re relasjoner \cite{xfst}.

Slik Xerox-verkt\o{}yene fungerer er prosjektet b\aa{}de en morfologisk
analysator og generator, ettersom alle nettverkene som produseres kan
kj\o{}res i begge retninger: b\aa{}de fra en leksikonform som
\form{amo+Verb+Pres+Subj+2S+Pass} og fra en overflateform som \w{ameris}. Jeg
har derfor valgt \aa{} skrive systemet som en morfologisk generator, selv om
m\aa{}let er \aa{} implementere en analysator, ettersom de fleste beskrivelser
av latinsk morfologi er strukturert p\aa{} denne m\aa{}ten, og det for meg er
kognitivt lettere \aa{} gi alle formene i et paradigme enn \aa{} enumerere
alle mulighetene for en gitt endelse.

Webgrensesnittet er et helt grunnleggende grensesnitt mot applikasjonens
fundamentale funksjoner, analyse og generering av latinsk morfologi,
implementert som et CGI-script i Python. Testene er ogs\aa{} skrevet i Python,
men genererer testdata i et format vanligvis brukt for \aa{} teste Perl-kode
(TAP). Jeg har valgt \aa{} gj\o{}re det p\aa{} denne m\aa{}ten ettersom TAP
har en sv\ae{}rt lav terskel for bruk, noe som har gjort at testene kan
implementeres med et minimum av arbeid.

\subsection{N\o{}dvendig programvare}
For \aa{} kompilere selve morfologidelen av applikasjonen trengs bare
XFST, og en implementasjon av \prog{make} for \aa{} bygge applikasjonen
(applikasjonen kan ogs\aa{} bygges manuelt). I tillegg trenger testsuiten
Python-grensesnittet mot XFST\footnote{http://link.to.code/}, biblioteket
\lib{PyTAP}\footnote{http://link.to.code/} og applikasjonen \prog{prove} fra
Perl-biblioteket \lib{Test::Harness} (installeres sammen med de fleste moderne
versjoner av Perl\note{Sjekke at dette stemmer.}). Webgrensesnittet krever
bare Python-grensesnittet mot XFST.

\subsection{Koden}
Koden er tilgjengelig fra TODO, og best\aa{}r av f\o{}lgende filer og
kataloger:

\note{Finn en bedre m\aa{}te \aa{} presentere dette p\aa{}.}
\begin{itemize}
\item[build/] Korte script som bygger og komponerer de forskjellige FSTene
p\aa{} riktig m\aa{}te.
\item[report/] \LaTeX{}-kildekoden til denne rapporten. % XXX: Fjerne fra den endelige leveransen?
\item[t/] Python-script som tester at koden virker som den skal.
\item[vim/] Script for editoren \prog{vim} for \aa{} f\aa{} syntaksmarkering
for \prog{lexc}-- og \prog{twolc}-kode.
\item[web/] Pythonkoden for webgrensesnittet.

\item[Makefile] Regler for \prog{make} for \aa{} bygge FSTene.
\item[TODO] Liste over ting som m\aa{} gj\o{}res. % XXX: Fjerne fra den endelige leveransen?
\item[README] Instruksjoner om bruk og installasjon av applikasjonen.
\item[nouns-lexc.txt] Leksikonspesifikasjon for substantiver.
\item[nouns-twolc.txt] Fonetiske regler for substantiver.
\item[verbs-lexc.txt] Leksikonspesifikasjon for verb.
\item[verbs-twolc.txt] Fonetiske regler for verb.
\end{itemize}

\subsection{Bruk av systemet}\note{Dette m\aa{} plasseres et annet sted.}
Et webgrensesnitt mot applikasjonen er tilgjengelig p\aa{}
\texttt{http://heim.ifi.uio.no/arnskj/cgi-bin/latin/analyse.cgi}, og
grensesnittet burde v\ae{}re ganske selvforklarende. En b\o{}yd form skrives
inn i tekstfeltet, og man klikker p\aa{} ``Analyser'' for \aa{} f\aa{} alle
mulige analyser av formen, eller man skriver inn en analyse og klikker p\aa{}
``Generer'' for \aa{} f\aa{} en liste over mulige overflateformer. Formatet
p\aa{} analysebeskrivelsene er fors\o{}kt holdt s\aa{} enhetlig som mulig,
p\aa{} formatet \form{lemma+Klasse[+Info\ldots]} og med rekkef\o{}lgen av
informasjonstaggene s\aa{} intuitivt som mulig. For verb er formatet
\form{lamma+Verb+Tempus+Modus+PersTall+Diatese}, for substantiver
\form{lemma\-+Noun\-+Kasus\-+Tall}, for adjektiver
\form{lemma+Adj+Grad\-+Kj\o{}nn+Kasus+Tall}, og for adverb
\form{lemma+Adv+Grad}.

\subsection{Nominalsystemet}\note{Skrive om, og beskrive den historiske
utviklingen i koden i stedet? Snakke om leksikon og slikt f\o{}rst, og s\aa{}
f\o{}ye til at det viste seg at fonologi ble nyttig?}
\label{deklinasjon}
Den latinske nominalmorfologien er forholdsvis enkel. I motsetning til for
eksempel nominalsystemet i sanskrit er det ingen lydendringer innad i
b\o{}yningsr\o{}ttene, og de f\aa{} tilfellene av lydendringer i m\o{}tet
mellom rot og endelse er enkle \aa{} hanskes med. Dette gir en oppbygning der
hver rot er bundet mot en b\o{}yningsklasse med sine karakteristiske endelser
for de enkelte formene, samt \'en fonetisk regel for \aa{} h\aa{}ndtere
r\o{}tter p\aa{} \w{-er} som mister \w{e} n\aa{}r det legges til en endelse.

For \aa{} lette implementasjonen har de fleste vekslinger av overflateformen
til roten blitt ansett for \aa{} v\ae{}re leksikalske (unntaket er bortfall av
\w{e} i \w{-er}). Rent lingvistisk hadde det kanskje v\ae{}rt riktigere \aa{}
modellere flere av disse fenomenene som regelrette med fonetiske regler, for
eksempel \w{reg-s} > \w{rex} og \w{flos-is} > \w{floris}. P\aa{} samme
m\aa{}te ender stammefinal vokal opp som endelsesinitial i f\o{}rste og fjerde
deklinasjon, slik at bortfallet av vokalen i dativ/ablativ flertall kan
h\aa{}ndteres i leksikon i stedet for med fonetiske regler. Som nevnt er den
ene fonetiske regelen i nominalsystemet for \aa{} h\aa{}ndtere
synkope\note{Synkope er riktig, no?} i \w{-er}-endelsen. Den fungerer ganske
enkelt slik at de r\o{}ttene som mister \w{e} med endelse har denne markert
som s\aa{}dan i leksikon og den fonetiske regelen fjerner den hvis det kommer
en endelse etter; opprinnelig var denne regelen mer generell, og synkoperte
alle forekomster av \w{-er} foran morfemgrense (unntatt i r\o{}ttene \w{puer},
\w{gener}, \w{socer}, \w{liber}, \w{vesper} og \w{signifer}), men det viser
seg at det finnes r\o{}tter p\aa{} \w{-er} som \emph{ikke} har synkope, for
eksempel \w{numerus}, og at synkopen er leksikalisert (se for\o{}vrig
\ref{diakroni} for mer diskusjon om dette).

Adjektivb\o{}yningen h\aa{}ndteres p\aa{} akkurat samme m\aa{}te som
substantivene, inkludert \w{e}-synkope. Derimot er avledningen av adverb noe
utradisjonelt implentert. Adverbavledning er implementert slik at de
forskjellige adverbformene er avledet fra adjektivets b\o{}yningsstamme i den
tilsvarende b\o{}yningsgraden. Dessverre inneb\ae{}rer dette at den genererte
analysen blir \form{lemma+Adj+Pos+Adv} eller tilsvarende, noe som helt klart
ikke er optimalt. Jeg har derfor lagt til en omskrivningsregel p\aa{} toppen
av substantivleksikonet som skriver om analyser p\aa{} denne formen til den
korrekte formen: \form{lemma+Adv+Pos}.

\subsection{Verbalsystemet}
\label{konjugasjon}
Verbalmorfologien er den mest interessante og kompliserte delen av systemet;
den er delt inn i to deler: en leksikondel, som inneholder mesteparten av
koden, og et sett fonetiske omskrivningsregler. Reglene i verbleksikonet tar
leksikonformer og skriver dem om til r\o{}tter etterfulgt av en streng
morfemer, for eksempel er resultatet for \w{amo} presens konjunktiv 2.~person
entall passiv \w{ama-e-ris}. Denne mellomformen behandles s\aa{} av de
fonetiske reglene som setter inn korrekte temavokaler og h\aa{}ndterer
bortfall av lyder; for eksempel omskrives \w{ama-o} til \w{amo} og \w{reg-ris}
til \w{regeris}. Tabellene \ref{endelser} og \ref{tempus} gir en oversikt over
de viktigste person-- og tempusendelsene, henholdsvis, som brukes i systemet.

% TODO: Prov aa flette disse tabellene bedre inn i teksten, saann at det er et
% avsnitt eller to mellom dem.
\begin{table}
\begin{center}
\begin{tabular}{|c|c|c|c|}
    \hline
       & \multicolumn{2}{c|}{Presens} & \multirow{2}{*}{Perfektum} \\ \cline{2-3}
       & Aktiv & Passiv & \\
    \hline
1.~sg. & \w{--m}   & \w{--r}    & \w{--i}     \\
2.~sg. & \w{--s}   & \w{--ris}  & \w{--isti}  \\
3.~sg. & \w{--t}   & \w{--tur}  & \w{--it}    \\
1.~pl. & \w{--mus} & \w{--mur}  & \w{--imus}  \\
1.~pl. & \w{--tis} & \w{--mini} & \w{--istis} \\
1.~pl. & \w{--nt}  & \w{--ntur} & \w{--erunt} \\
    \hline
\end{tabular}
\caption{Personendelsene}
\label{endelser}
\end{center}
\end{table}

\begin{table}
\begin{center}
\begin{tabular}{|c|c|c|}
    \hline
                  & Indikativ   & Konjunktiv \\
    \hline
Presens           & $\emptyset$ & \w{--e} (1), \w{--a} (2, 3, 4) \\
Imperfektum       & \w{--ba} & \w{--re} \\
Futurum           & \w{--b} (1, 2), \w{--a} (3, 4) & --- \\
    \hline
Perfektum         & $\emptyset$ & \w{--eri} \\
Pluskvamperfektum & \w{--era} & \w{--isse} \\
Futurum exactum   & \w{--eri} & ---        \\
    \hline
\end{tabular}
\caption{Tempussuffiksene}
\label{tempus}
\end{center}
\end{table}

\note{Faa med at 3B verb noteres med -I i leksikon for aa merke dem som
spesielle.}

\note{De fonetiske reglene er oppdatert siden dette ble skrevet. Sjekk mot
verbs-twolc.txt!}
\paragraph{\code{a:0 <=> \_ \%-: [ o | e ]}} Denne regelen h\aa{}ndterer to
tilfeller hvor stammefinal \w{--a} faller bort: foran \w{--o} i 1.~person
entall presens indikativ og foran suffikset for presens konjunktiv.

\paragraph{\code{\%-:i <=> Cons \_ [ s | t | m ]}} Temavokal i
konsonantstammer er \w{i} foran \w{s}, \w{t} og \w{m}.

\paragraph{\code{\%-:e <=> Cons \_ r}} Temavokal i konsonantstammer er
\w{e} foran \w{r}.

\paragraph{\code{\%-:u <=> Cons \_ n}} Temavokal i konsonantstammer er
\w{u} foran \w{n}.

\paragraph{\code{I:e <=> \_ \%-: r}} Som foreg\aa{}ende regel, men for
stammer p\aa{} kort \w{i}.

\paragraph{\code{I:0 <=> \_ \%-: i .\#.}} Stammefinal kort \w{i} faller bort
foran endelsen \w{--i} (presens infinitiv passiv).

Disse reglene beskriver mer eller mindre generelle regler. De resterende
reglene h\aa{}ndterer den valgfrie \w{v}-synkopen i perfektumssystemet.

\paragraph{\code{v:0 => \ldots}} \w{v} mellom vokal og morfemgrense kan i
enkelte tilfeller falle bort (se \S 41.3 i \citeN{eitrem} og \citeN{ernout},
s.~209-212).

\paragraph{\code{v:0 /<= a \_ \%-: ? r e .\#.}} Bortfall av \w{v} m\aa{}
ikke skje n\aa{}r resultatet vil kunne forveksles med en infinitiv.

\paragraph{\code{e:0 <=> a v:0 \%-: \_}} \w{e} etter bortfalt \w{v} foran
\w{a} faller ogs\aa{} bort.

\paragraph{\code{i:0 <=> a v:0 \%-: \_ ; i v:0 \%-: \_ s}} \w{i} etter
bortfalt \w{v} foran \w{a}, eller foran \w{i} og etter bortfalt \w{v} foran
\w{i}, faller ogs\aa{} bort.

\subsection{Verifisering}
For \aa{} kontrollere at endringer i en del av koden ikke \o{}delegger noe
annet har jeg skrevet en del tester. Testene er implementert som
Python-programmer som genererer output etter protokollen ``Test Anything
Protocol'' (TAP). Programmet \prog{prove} kj\o{}rer s\aa{} alle
testprogrammene, tolker resultatene og skriver ut statistikker. Testene er
skrevet slik at for hvert par av leksikonform og korrekte former blir de
mulige overflateformene i f\o{}lge reglene generert, og programmet sjekker at:
1) reglene genererer samme antall former som er spesifisert i testprogrammet
og 2) alle formene i testprogrammet blir generert. Totalt blir det utf\o{}rt
5638 tester, for 2892 overflateformer \note{Disse tallene m\aa{} oppdateres!}
(inklusive tilfeller der forskjellige leksikonformer har samme overflateform).
Testene er i stor grad generert ved hjelp av klipp-og-lim og erstatninger ved
hjelp av regul\ae{}re uttrykk.

\section{Synkroni vs.~diakroni}
\label{diakroni}
En morfologisk modell er synkron av natur, men det kan kan ogs\aa{} v\ae{}re
interessant \aa{} se hvordan den synkrone modellen passer med den diakrone
utviklingen av det morfologiske systemet.\note{Ting som m\aa{} med:
\begin{itemize}
\item \w{-er}-synkope. Leksikalisert. Hvorfor?
\end{itemize}
}

\section{Konklusjon}\note{Bedre tittel!}
\note{Ta med ideen om \aa{} simulere historiske regler med
toniv\aa{}formalismen? Det er en riktig festlig ide, og er ikke helt p\aa{}
jordet irrelevant heller, gitt den siste delen med sammenligning av synkroni
og diakroni\ldots}

\clearpage
\bibliographystyle{norchicago}
\bibliography{report}{}

\tableofcontents
\listoftables

\end{document}
