\documentclass{article}

\usepackage[norsk]{babel}
%\usepackage{cite}
\usepackage{chicago}
\usepackage[T1]{fontenc}

\title{Toniv\aa{}morfologi for latin}
\author{Arne Skj\ae{}rholt}

\begin{document}
\maketitle
% XXX: Move TOC (and stuff like list of figures/tables if they crop up as
% well) to the end of the document instead of the top?
\tableofcontents

\section{Introduksjon}
\subsection{Form\aa{}l}
Det \aa{} implementere en komplett toniv\aa{}modell er ikke en triviell
oppgave, det er blant annet skrevet hovedfagsoppgaver der m\aa{}let er nettopp
det, se \cite{bondihoved} og \cite{french}, s\aa{} m\aa{}let for dette
prosjektet vil v\ae{}re \aa{} implementere en delmengde av det latinske
formverket som er s\aa{} interessant som mulig; der interessant er definert
slik at s\aa{} mange implementasjonstekniske problemer som mulig blir
eksponert. Uregelmessige former, selv om de kanskje i praktisk hensyn er
veldig interessante, er for eksempel ikke s\ae{}rlig inspirerende i denne
sammenhengen ettersom implementasjonen stort sett kan oppsummeres som ``for
denne roten ser denne formen slik ut''.

Forh\aa{}pentligvis vil den endelige implementasjonen inkludere b\aa{}de
verbal-- og nominalmorfologi der alle regelrette former av regelmessige
r\o{}tter er inkludert.

\subsection{Planen}
The plan of attack is as follows: first read up on the tools to be used and
the theory involved. Then, start with implementing a subset of verbal
morphology; this will probably be the active present system of regular verbs
of the first class, then the perfect system. Experiences from this should give
some ideas on how to structure different inflectional stems for a single
lexical entry, and from this it will hopefully be relatively easy to implement
regular nominal morphology. After this we may either proceed to pronominal or
adjectival morphology, or return to the verbal morpholgy and add more verb
classes, passives and non-finite forms of the verb.

%Certain factors may alter this plan, however. The current corpus should be
%consulted to generate statistics of which parts of the morphology are more
%frequent and thus more likely to be useful to implement first. Also, the users
%of the system should be asked about which parts they would prefer to be
%implemented first.
% TODO: Actually do this!

\section{Implementasjonen}
\subsection{Verkt\o{}yene}
Selve analysatoren er skrevet med Xerox-verkt\o{}ykassen for endelig
tilstandsmorfologi, dokumentert i \citeNP{xfst}.

%\begin{itemize}
%    \item XFST \cite{xfst}
%    \item Lewis and Short from Perseus for lexicon data? % TODO: Find reference
%\end{itemize}

\subsection{Nominalsystemet}
Det viser seg at nominalmorfologien er enkel nok til at det holder \aa{} ha et
leksikon med r\o{}tter koblet til b\o{}yningsklasse der de forskjellige
b\o{}yningsklassene deler ut egnede endelser til de forskjellige kasus.

Rent lingvistisk hadde kanskje en tiln\ae{}rming lik den som er benyttet for
konjugasjonen (se seksjon \ref{conjugation}) v\ae{}rt riktigere, men for \aa{}
lette implementasjonen (og for min sjelefred) er en forenklet metode
foretrukket. Den viktigste konsekvensen av dette er at enhver form for
veksling mellom b\o{}yningsstammen i de oblike kasus og formen i nominativ
entall i tredje deklinasjon er ansett som leksikalsk, selv om den kan anses
som regelmessig, \`a la \emph{reg-s} > \emph{rex}. P\aa{} samme m\aa{}te ender
stammefinal vokal opp som endelsesinitiell vokal i stedet i
vokaldeklinasjonene p\aa{} grunn av stammevokalens bortfall i enkelte former;
for eksempel er stammen til \emph{rosa} ansett for \aa{} v\ae{}re \emph{ros--}
slikt at endelsen i dat/ablativ flertall kan v\ae{}re \emph{--is} uten flere
regler for \aa{} h\aa{}ndtere tapet av den stammefinale \emph{--a}.

\subsection{Verbalsystemet}
\label{conjugation}
% TODO

\section{Conclusion} % TODO: Better title!

% TODO: Find references for grammars. Eitrem, Ernout & Thomas, the other one. The Swedish one.
\bibliographystyle{chicago} % TODO: Better citation style?
\bibliography{report}{}

\end{document}
