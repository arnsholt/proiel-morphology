\documentclass{article}

\usepackage[norsk]{babel}
%\usepackage{cite}
\usepackage{chicago}
\usepackage[T1]{fontenc}
\usepackage{footnote}
\usepackage{multirow}
\usepackage{verbatim}

\title{Toniv\aa{}morfologi for latin}
\author{Arne Skj\ae{}rholt}

\begin{document}
\maketitle
% XXX: Move TOC (and stuff like list of figures/tables if they crop up as
% well) to the end of the document instead of the top?
\tableofcontents

\section{Introduksjon}
\subsection{Form\aa{}l}
Det \aa{} implementere en komplett toniv\aa{}modell er ikke en triviell
oppgave, det er blant annet skrevet hovedfagsoppgaver der m\aa{}let er nettopp
det, se \cite{bondihoved} og \cite{french}, s\aa{} m\aa{}let for dette
prosjektet har v\ae{}rt \aa{} implementere en delmengde av det latinske
formverket som er s\aa{} interessant som mulig; der interessant er definert
slik at s\aa{} mange implementasjonstekniske problemer som mulig blir
eksponert. Uregelmessige former, selv om de kanskje rent praktisk sett er
veldig interessante, er for eksempel ikke s\ae{}rlig inspirerende i denne
sammenhengen ettersom implementasjonen stort sett kan oppsummeres som ``for
denne roten ser denne formen slik ut''.

\subsection{Planen}
Den planlagte fremgangsm\aa{}ten, som stort sett korresponderer med den
faktiske gangen i arbeidet med implementasjonen var \aa{} begynne i det
sm\aa{} med regelrette substantiver for \aa{} bli kjent med verkt\o{}yene og
gi grunnleggende f\o{}ringer p\aa{} hvordan de kan integreres med standard
UNIX-verkt\o{}y som \texttt{make} for \aa{} automatisere
kompilasjonsprosessen. Etter dette sto verbalsystemet for tur. Siden
verbalparadigmet er den mest kompliserte delen av den latinske morfologien
ans\aa{} jeg det som viktig \aa{} utforske denne delen av systemet for \aa{}
finne ut hvor kompliserte regler som trengtes. Etter dette var planen \aa{}
implementere det som gjensto av de regelrette verbal-- og
substantivparadigmene.

\subsection{Funksjonalitet}
% TODO: Dette maa skrives naar systemet er ferdig(ere).

\section{Implementasjonen}
\subsection{Kilder} % XXX: Bedre tittel!
Grunnlaget for forml\ae{}ren er i all hovedsak tatt fra de relevante
paragrafene i \citeNP{eitrem}, men med en del detaljer fylt ut av
\citeNP{ernout}; sistnevnte var s\ae{}rlig opplysende i enkelte detaljer
omkring perfektumsformer med synkope av \emph{--v}.

\subsection{Verkt\o{}yene}
Selve analysatoren er skrevet med Xerox-verkt\o{}ykassen for endelig
tilstandsmorfologi, dokumentert i \citeNP{xfst}; prosjektet benytter seg av
\texttt{lexc} til leksikon og \texttt{twolc} for fonetiske regler.
\texttt{lexc} spesifiserer regul\ae{}re relasjoner som h\o{}yrerekursive
grammatikker og \texttt{twolc} spesifiserer relasjoner i form av tegn som er
relatert, og i hvilke kontekster de kan, m\aa{} eller ikke kan opptre.

Slik Xerox-verkt\o{}yene fungerer er prosjektet b\aa{}de en morfologisk
analysator og generator, ettersom alle nettverkene som produseres kan
kj\o{}res i begge retninger: b\aa{}de fra en leksikonform som
\emph{amo+Verb+Pres+Subj+2S+Pass} og fra en overflateform som \emph{ameris}.

\subsection{Nominalsystemet}
Det viser seg at nominalmorfologien er enkel nok til at det holder \aa{} ha et
leksikon med r\o{}tter koblet til b\o{}yningsklasse der de forskjellige
b\o{}yningsklassene deler ut egnede endelser til de forskjellige kasus.

Rent lingvistisk hadde kanskje en tiln\ae{}rming lik den som er benyttet for
konjugasjonen (se seksjon \ref{conjugation}) v\ae{}rt riktigere, men for \aa{}
lette implementasjonen (og for min sjelefred) er en forenklet metode
foretrukket. Den viktigste konsekvensen av dette er at enhver form for
veksling mellom b\o{}yningsstammen i de oblike kasus og formen i nominativ
entall i tredje deklinasjon er ansett som leksikalsk, selv om den kan anses
som regelmessig, \`a la \emph{reg-s} > \emph{rex}. P\aa{} samme m\aa{}te ender
stammefinal vokal opp som endelsesinitiell vokal i stedet i
vokaldeklinasjonene p\aa{} grunn av stammevokalens bortfall i enkelte former;
for eksempel er stammen til \emph{rosa} ansett for \aa{} v\ae{}re \emph{ros--}
slikt at endelsen i dat/ablativ flertall kan v\ae{}re \emph{--is} uten flere
regler for \aa{} h\aa{}ndtere tapet av den stammefinale \emph{--a}.

\subsection{Verbalsystemet}
\label{conjugation}
Verbalmorfologien er den mest interessante og kompliserte delen av systemet;
den er delt inn i to deler: en leksikondel, som inneholder mesteparten av
koden, og et sett fonetiske omskrivningsregler. Reglene i verbleksikonet tar
leksikonformer og skriver dem om til r\o{}tter etterfulgt av en streng
morfemer, for eksempel er resultatet for \emph{amo} presens konjunktiv
2.~person entall passiv \emph{ama-e-ris}. Denne mellomformen behandles s\aa{}
av de fonetiske reglene som setter inn korrekte temavokaler og h\aa{}ndterer
bortfall av lyder; for eksempel omskrives \emph{ama-o} til \emph{amo} og
\emph{reg-ris} til \emph{regeris}. Tabellene \ref{endelser} og \ref{tempus}
gir en oversikt over de viktigste person-- og tempusendelsene, henholdsvis,
som brukes i systemet.

\begin{table}[hp]
\begin{center}
\begin{tabular}{|c|c|c|c|}
    \hline
       & \multicolumn{2}{c|}{Presens} & \multirow{2}{*}{Perfektum} \\ \cline{2-3}
       & Aktiv & Passiv & \\
    \hline
1.~sg. & \emph{--m}   & \emph{--r}    & \emph{--i}       \\
2.~sg. & \emph{--s}   & \emph{--ris}  & \emph{--isti}    \\
3.~sg. & \emph{--t}   & \emph{--tur}  & \emph{--it}      \\
1.~pl. & \emph{--mus} & \emph{--mur}  & \emph{--imus}    \\
1.~pl. & \emph{--tis} & \emph{--mini} & \emph{--istis}   \\
1.~pl. & \emph{--nt}  & \emph{--ntur} & \emph{--erunt}   \\
    \hline
\end{tabular}
\caption{Personendelsene}
\label{endelser}
\end{center}
\end{table}

\begin{table}[hp]
\begin{center}
\begin{tabular}{|c|c|c|}
    \hline
                  & Indikativ   & Konjunktiv \\
    \hline
Presens           & $\emptyset$ & \emph{--e} (1), \emph{--a} (2, 3, 4) \\
Imperfektum       & \emph{--ba} & \emph{--re} \\
Futurum           & \emph{--b} (1, 2), \emph{--a} (3, 4) & --- \\
    \hline
Perfektum         & $\emptyset$ & \emph{--eri} \\
Pluskvamperfektum & \emph{--era} & \emph{--isse} \\
Futurum exactum   & \emph{--eri} & ---        \\
    \hline
\end{tabular}
\caption{Tempussuffiksene}
\label{tempus}
\end{center}
\end{table}

\paragraph{\texttt{a:0 <=> \_ \%-: [ a | e ]}} Denne regelen h\aa{}ndterer to
tilfeller hvor stammefinal \emph{--a} faller bort: foran \emph{--o} i
1.~person entall presens indikativ og foran suffikset for presens konjunktiv.

\paragraph{\texttt{\%-:i <=> Cons \_ [ s | t | m ]}} Temavokal i
konsonantstammer er \emph{i} foran \emph{s}, \emph{t} og \emph{m}.

\paragraph{\texttt{\%-:e <=> Cons \_ r}} Temavokal i konsonantstammer er
\emph{e} foran \emph{r}.

\paragraph{\texttt{\%-:u <=> Cons \_ n}} Temavokal i konsonantstammer er
\emph{u} foran \emph{n}.

\paragraph{\texttt{I:e <=> \_ \%-: r}} Som foreg\aa{}ende regel, men for
stammer p\aa{} kort \emph{i}.

\paragraph{\texttt{I:0 <=> \_ \%-: i .\#.}} Stammefinal kort \emph{i} faller
bort foran endelsen \emph{--i} (presens infinitiv passiv).

Disse reglene beskriver mer eller mindre generelle regler. De resterende
reglene h\aa{}ndterer den valgfrie \emph{v}-synkopen i perfektumssystemet.

\paragraph{\texttt{v:0 => \ldots}} \emph{v} mellom vokal og morfemgrense kan i
enkelte tilfeller falle bort (se \S 41.3 i \citeNP{eitrem} og \citeNP{ernout},
s.~209-212).

\paragraph{\texttt{v:0 /<= a \_ \%-: ? r e .\#.}} Bortfall av \emph{v} m\aa{}
ikke skje n\aa{}r resultatet vil kunne forveksles med en infinitiv.

\paragraph{\texttt{e:0 <=> a v:0 \%-: \_}} \emph{e} etter bortfalt \emph{v}
foran \emph{a} faller ogs\aa{} bort.

\paragraph{\texttt{i:0 <=> a v:0 \%-: \_ ; i v:0 \%-: \_ s}} \emph{i} etter
bortfalt \emph{v} foran \emph{a}, eller foran \emph{i} og etter bortfalt
\emph{v} foran \emph{i}, faller ogs\aa{} bort.

\section{Synkroni vs.~diakroni}
En morfologisk modell er synkron av natur, men det kan kan ogs\aa{} v\ae{}re
interessant \aa{} se hvordan den synkrone modellen passer med den diakrone
utviklingen av det morfologiske systemet.

\section{Konklusjon} % TODO: Better title!

\clearpage
\bibliographystyle{norchicago}
\bibliography{report}{}

\end{document}
